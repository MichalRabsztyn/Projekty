\documentclass[12pt,a4paper,oneside]{article}

\usepackage{amsmath,amssymb}
\usepackage[utf8]{inputenc}                                      
\usepackage[OT4]{fontenc}                                
\usepackage[polish]{babel}                           
\selectlanguage{polish}
\usepackage{indentfirst} 
\usepackage[dvips]{graphicx}
\usepackage{tabularx}
\usepackage{color}
\usepackage{hyperref} 
\usepackage{fancyhdr}
\usepackage{listings}
\usepackage{booktabs}
\usepackage{ifpdf}
\usepackage{mathtext}
\usepackage{lmodern}
\usepackage{filecontents}
\usepackage{ifthen}


\usepackage{tikz}
\usetikzlibrary{calc,shapes.multipart,chains,arrows}


\newcounter{nextYear}
\setcounter{nextYear}{\the\year}
\stepcounter{nextYear}

% rozszerzenie nieco strony
\setlength{\topmargin}{-1cm} \setlength{\textheight}{20.5cm}
\setlength{\textwidth}{17cm} \addtolength{\hoffset}{-1.5cm}
\setlength{\parindent}{0.5cm} \setlength{\footskip}{2cm}
\linespread{1.2} % odstep pomiedzy wierszami


%%%% ZYWA PAGINA %%%%%%%%%%%
\newcommand{\tl}[1]{\textbf{#1}} 
\pagestyle{fancy}
\renewcommand{\sectionmark}[1]{\markright{\thesection\ #1}}
\fancyhf{} % usuwanie bieżących ustawień
\fancyhead[LE,RO]{\small\bfseries\thepage}
\fancyhead[LO]{\small\bfseries\rightmark}
\fancyhead[RE]{\small\bfseries\leftmark}
\renewcommand{\headrulewidth}{0.5pt}
\renewcommand{\footrulewidth}{0pt}
\addtolength{\headheight}{0.5pt} % pionowy odstęp na kreskę
\fancypagestyle{plain}{%
\fancyhead{} % usuń p. górne na stronach pozbawionych numeracji
\renewcommand{\headrulewidth}{0pt} % pozioma kreska
}

%%%%%   LISTINGI %%%%%%%%
% ustawienia listingu programow

\lstset{%
language=C++,%
commentstyle=\textit,%
identifierstyle=\textsf,%
keywordstyle=\sffamily\bfseries, %
%captionpos=b,%
tabsize=3,%
frame=lines,%
numbers=left,%
numberstyle=\tiny,%
numbersep=5pt,%
breaklines=true,%
morekeywords={pWezel,Wezel,string,ref,params_result},%
escapeinside={(*@}{@*)},%
%basicstyle=\footnotesize,%
%keywords={double,int,for,if,return,vector,matrix,void,public,class,string,%
%float,sizeof,char,FILE,while,do,const}
}
%%%%%%%%%%%%%%%%%%%%%%%%%%%%%%%%%%%%%%%%%%%%%%%%%%%%%%%%%%%%%%%%%%%%%%%

%%%%%%%%%  NOTKI NA MARGINESIE %%%%%%%%%%%%%
% mala zmiana sposobu wyswietlania notek bocznych
\let\oldmarginpar\marginpar
\renewcommand\marginpar[1]{%
  {\linespread{0.85}\normalfont\scriptsize%
\oldmarginpar[\hspace{1cm}\begin{minipage}{3cm}\raggedleft\scriptsize\color{black}\textsf{#1}\end{minipage}]%    left pages
{\hspace{0cm}\begin{minipage}{3cm}\raggedright\scriptsize\color{black}\textsf{#1}\end{minipage}}% right pages
}%
}
% % % % % % % % % % % % % % % % % % % % % % % % % % % % % % % %

%%%% WYSWIETLANIE AKTUALNEGO ROKU AKADEMICKIEGO %%%%%%%%%%%
\newcounter{rok}
\newcommand{\rokakademicki}{%
   \setcounter{rok}{\number\year}%
   \ifthenelse{\number\month<10}%
   {\addtocounter{rok}{-1}}% rok akademicki zaczal sie w pazdzierniku poprzedniego roku
   {}%                       rok akademicki zaczyna sie w pazdzierniku tego roku
   \arabic{rok}/\addtocounter{rok}{1}\arabic{rok}
}
%%%%%%%%%%%%%%%%%%%%%%%%%%%%%%%%%%%%%%%


%%%% LISTA UWAG %%%%%%%%%
\usepackage{color}
\definecolor{brickred}      {cmyk}{0   , 0.89, 0.94, 0.28}

\makeatletter \newcommand \kslistofremarks{\section*{Uwagi} \@starttoc{rks}}
\newcommand\l@uwagas[2]
{\par\noindent \textbf{#2:} %\parbox{10cm}
   {#1}\par} \makeatother


\newcommand{\ksremark}[1]{%
   {{\color{brickred}{[#1]}}}%
   \addcontentsline{rks}{uwagas}{\protect{#1}}%
}

\newcommand{\comma}{\ksremark{przecinek}}
\newcommand{\nocomma}{\ksremark{bez przecinka}}
\newcommand{\styl}{\ksremark{styl}}
\newcommand{\ortografia}{\ksremark{ortografia}}
\newcommand{\fleksja}{\ksremark{fleksja}}
\newcommand{\pauza}{\ksremark{pauza `--', nie dywiz `-'}}
\newcommand{\kolokwializm}{\ksremark{kolokwializm}}
\newcommand{\cytowanie}{\ksremark{cytowanie}}

%%%%%%%%%%%%%%%%%%%%%%%%%
%%%%%%%%%%%%%%%%%%%%%%%%%
%%%%%%%%%%%%%%%%%%%%%%%%%
%%%%%%%%%%%%%%%%%%%%%%%%%
%%%%%%%%%%%%%%%%%%%%%%%%%
%%%%%%%%%%%%%%%%%%%%%%%%%
%%%%%%%%%%%%%%%%%%%%%%%%%
%%%%%%%%%%%%%%%%%%%%%%%%%
%%%%%%%%%%%%%%%%%%%%%%%%%
%%%%%%%%%%%%%%%%%%%%%%%%%
%%%%%%%%%%%%%%%%%%%%%%%%%
%%%%%%%%%%%%%%%%%%%%%%%%%



% autor:
\fancyhead[RE]{\small\bfseries Michał Rabsztyn} % autor sprawozdania



%%%%%%%%%%% NO I ZACZYNA SIE SPRAWOZDANIE %%%%%%%%%%%

\begin{document}
\frenchspacing
\thispagestyle{empty}
\begin{center}
{\Large\sf Politechnika Śląska   % Alma Mater

Wydział Automatyki, Elektroniki i Informatyki

}

\vfill

 

\vfill\vfill

{\Huge\sffamily\bfseries Programowanie Komputerów\par}  

\vfill\vfill

{\LARGE\sf Koło Fortuny}   


\vfill \vfill\vfill\vfill

%%%%%%%%%%%%%%%%%%%%%%%%%%%%





\begin{tabular}{ll}
	\toprule
	autor                       & Michał Rabsztyn        	   \\
	prowadzący                  & dr inż. Bożena Wieczorek   \\
	rok akademicki              & \rokakademicki        	   \\
	kierunek                    & informatyka           	   \\
	rodzaj studiów              & SSI                    	   \\
	semestr                     & 2                      	   \\
	termin laboratorium         & czwartek, 08:30 -- 10:00 \\
	sekcja                      & 12                   	   \\
	termin oddania sprawozdania & 2020-08-26           	   \\
	\bottomrule
	                            &
\end{tabular}

\end{center}
%%% koniec strony  tytulowej

%%%%%%%%%%%%%%%%%%%%%%%%%%%%%%%%%%%%%%%%%%%%%%%%%%%%%%%%%%%%%%%%%%%%%%%%%
%\cleardoublepage
%%%%%%%%%%%%%%%%%%%%%%%%%%%%%%%%%%%%%%%%%%%%%%%%%%%%%%%%%%%%%%%%%%%%%%%%%

%%%%%%%%%%%%%%%%%%%%%%%%%%%%%%%%%%%%%%%%%%%%%%%%%%%%%%%%%%%%%%%%%%%%%%%%%
\section{Treść zadania}
\textbf{Wymagania}
\newline
\begin{tabular}{ll}
\texttt{o}  & użyto języka C,\\
\texttt{o}  & program podzielony jest na pliki źródłowe i nagłówkowe,\\
\texttt{o}  & funkcje dokumentowane są w doxygenie,\\
\texttt{o}  & na platformie znajduje się sprawozdanie (.pdf).\\
\end{tabular}
\newline 
\newline
\textbf{8.} Gra "Koło fortuny'' - kilka kategorii haseł.
\newline

%%%%%%%%%%%%%%%%%%%%%%%%%%%%%%%%%%%%%%%%%%%%%%%%%%%%%%%%%%%%%%%%%%%%%%%%%
\section{Analiza zadania}
Zagadnienie przedstawia problem przyjmowania danych wprowadzanych przez użytkowanika, porównywania ich ze wzorem odczytanym z pliku i zwracania informacji o ich poprawności.




\subsection{Algorytmy}
Program najpierw porównuje wprowadzony znak ze wskazanym zakresem znaków ascii, a następnie z wszystkimi literami znajdującymi się we wzorze hasła. Ilość porównań zależy od długości hasła.     

\begin{figure}
\centering
\begin{tikzpicture}
%\draw [help lines] grid (7,8);
\node[rectangle, draw] (A) at (5,10) {$main()$};
	\node[rectangle, draw] (B) at (0,8) {$obslugaArgumentow()$};
	\node[rectangle, draw] (C) at (2,7) {$wypiszZasady()$};
	\node[rectangle, draw] (D) at (4,6) {$czytajZPliku()$};
	\node[rectangle, draw] (E) at (6,5) {$obslugaZnaku()$};
	\node[rectangle, draw] (F) at (7.5,6) {$zakryjHaslo()$};
	\node[rectangle, draw] (G) at (10.2,7) {$wypiszAktualnyStanTablicy()$};
	\node[rectangle, draw] (H) at (12,8) {$wprowadzanieZnaku()$};
	\node[rectangle, draw] (I) at (14,9) {$wygrana()$};
		\node[rectangle, draw] (J) at (6.5,3) {$znakJestLitera()$};
		\node[rectangle, draw] (K) at (1,3) {$zgadywanieHasla()$};	
			\node[rectangle, draw] (L) at (0,1) {$liczWystapieniaLiteryWHasle()$};
			\node[rectangle, draw] (M) at (4,0) {$koloFortuny()$};
			\node[rectangle, draw] (N) at (8.5,0) {$wypiszAktualnyStanTablicy()$};
			\node[rectangle, draw] (O) at (13,1) {$sprawdzCzyWygrana()$};
	
\draw[>=latex,->] (A) -- (B);
\draw[>=latex,->] (A) -- (C);
\draw[>=latex,->] (A) -- (D);
\draw[>=latex,->] (A) -- (E);
\draw[>=latex,->] (A) -- (F);
\draw[>=latex,->] (A) -- (G);
\draw[>=latex,->] (A) -- (H);
\draw[>=latex,->] (A) -- (I);
\draw[>=latex,->] (E) -- (J);
\draw[>=latex,->] (E) -- (K);
\draw[>=latex,->] (J) -- (L);
\draw[>=latex,->] (J) -- (M);
\draw[>=latex,->] (J) -- (N);
\draw[>=latex,->] (J) -- (O);



\end{tikzpicture}
\caption{Schemat zależności pomiędzy funkcjami}
\label{fig:lista}
\end{figure} 

%%%%%%%%%%%%%%%%%%%%%%%%%%%%%%%%%%%%%%%%%%%%%%%%%%%%%%%%%%%%%%%%%%%%%%%%%
\section{Specyfikacja zewnętrzna}
\label{sec:sp:zewnetrzna}
Program jest uruchamiany z linii poleceń. Należy przekazać do programu nazwę pliku wejściowego po przełączniku -h. Opcjonalnie można podać przełącznik -z, co wyświetlizasady gry, np.
\begin{verbatim}
program -z -h hasla.txt
program -h hasla.txt -z
program -h hasla.txt
\end{verbatim}
Przełączniki mogą być podane w dowolnej kolejności. Uruchomienie programu ze zbyt małą lub zbyt dużą liczbą parametrów 
\begin{verbatim}
program
program -h hasla.txt -z zasady
program -z
\end{verbatim}
powoduje wyświetlenie pomocy.  



%%%%%%%%%%%%%%%%%%%%%%%%%%%%%%%%%%%%%%%%%%%%%%%%%%%%%%%%%%%%%%%%%%%%%%%%%
\section{Specyfikacja wewnętrzna}\label{sec:sp-wew}
Program został zrealizowany zgodnie z paradygmatem strukturalnym.  
W programie rozdzielono interfejs od logiki aplikacji.

%\lstinline|

\subsection{Ogólna struktura programu}
Program rozpoczyna działanie od sprawdzenia poprawności podanych parametrów. Jeżeli wszystko zostało podane poprawnie, program może kontynuować działanie. Losowana jest liczba z zakresu 1-(liczba haseł), a najstępnie z pliku odczytywane jest hasło, które zapisywane jest jako wzór. Tworzona jest tablica będąca reprezantacją odgadniętych i nieodgadniętych liter ze wzoru hasła. Dopóki hasło nie zostanie w pełni odgadnięte, użytkownik proszony jest o podanie znaku z klawiatury. Funkcja obslugaZnaku decyduje czy wprowadzony znak jest literą, znakiem zapytania czy nieobsługiwanym znakiem. Gdy rozpoznana zostanie litera, funkcja znakJestLitera() decyduje, w zależności od tego czy jest to samogłoska, czy spółgłoska i odpowiednio odejmuje lub dodaje punkty. Liczba dodanych punktów zależy od wartości wylosowanej w funkcji koloFortuny i liczby wystąpień litery we wzorze hasła. Użytkownik może podać w dowolnym momencie całe hasło wprowadzając najpierw "?", a następnie wpisując odpowiedni ciąg liter.



\subsection{Szczegółowy opis typów i funkcji}

Szczegółowy opis typów i funkcji zawarty jest w załączniku.

 

\section{Testowanie}
	Program został przetestowany na pliku hasla.txt zawierającym łącznie 50 haseł z 10 różnych kategorii.
	Program został sprawdzony pod kątem wycieków pamięci.


\section{Wnioski}
Dla urozmaicenia rozgrywki można dodać fukcjonalność bonusów oraz większe urozmaicenie kategorii haseł wraz z podziałem na poziomy trudności.
 
\cleardoublepage

\rule{0cm}{0cm}

\vfill

\begin{center}
\Huge\bfseries Dodatek\\Szczegółowy opis typów i~funkcji\par
\end{center}

\vfill 

\rule{0cm}{0cm}

\end{document}
% Koniec wieńczy dzieło.
